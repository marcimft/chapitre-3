\documentclass{report}

\usepackage{amsthm,amsmath,amssymb,mathrsfs,amsfonts,mathtools}
\usepackage{hyperref}
\usepackage{pgf, tikz}
\usepackage[parfill]{parskip}
\usepackage{graphicx}
\usepackage{subcaption}
\graphicspath{{figures/}}
\newcommand{\dd}{\mathrm{d}}
%\modulolinenumbers[5]


\begin{document}

\section{Motivation}
\subsection{loth 2009}
The pulse wave velocity (PWV) in a compliant vessel increases as
wall stiffness increases and has been of interest since arterial stiffness
is thought to be a risk factor for arterial disease
The pulse wave velocity (PWV) in a compliant vessel increases as
wall stiffness increases and has been of interest since arterial stiffness
is thought to be a risk factor for arterial disease [1]. Craniospinal dis
orders,
such as Chiari malformation, syringomyelia, and others, are
thought to be linked with overall cerebrospinal fluid (CSF) system compliance, and hence, PWV measurements would also be of inter
est \cite{william_1976} \cite{martin loth 2005}
In vivo PWV has been quantified through invasive measurement
of pressure in the CSF system \cite{williams_1976}. However, researchers have
found it difficult to obtain CSF PWV noninvasively due to a lack of
accessibility to the skull and spinal vertebrae..


\section{Review}

\subsection{loth 2011 very interesting, a kind of a review}

As velocity wave
speed is known to relate to the material properties of
a flow conduit, it may provide an estimate of tissue
stiffness in the craniospinal system when measured in
the spinal subarachnoid space. As increased tissue
stiffness is known to result from increased pressures,
this technique may be useful in quantifying stiffness
differences in the spinal subarachnoid space of
healthy subjects

Given the inconsistent results of magnetic resonance
imaging-based studies, engineers have sought to
better understand differences in the craniospinal
system of patients with type I Chiari malformation
by analyzing the hydrodynamics present in cerebrospinal
fluid flow that are not directly measurable
by imaging. The ultimate goal of these engineering
studies has been to gain insight into the biomechanical
nature of the disease. This may be accomplished
through the use of computational and experimental
tools, which seek to improve diagnostic methods and
surgical planning.
Computational fluid dynamics and in vitro flow
models allow non-invasive analysis of the hydrodynamic
environment in the craniospinal system.
Computational fluid dynamics simulations have been
helpful in describing the hemodynamics of blood flow
in arteries and veins44,45 where the shear stresses
created by blood flow have been shown to be
important in the development of arterial diseases
Both computational and in vitro modeling of
cerebrospinal fluid flow typically begins with a model
of the geometry of interest, either idealized or
reconstructed from anatomic magnetic resonance
images using solid modeling software, and flow data
obtained from one-dimensional phase-contrast imaging
measurements. Computational fluid dynamics
models approximate hydrodynamic variables, such as
pressure and fluid velocity, in the flow field by
utilizing the Navier–Stokes equations to numerically
simulate cerebrospinal fluid flow. The computational
fluid models can be coupled with computational solid
models to understand the solid stresses within the
neural tissue as a result of the fluid structure
interaction.39,40

Computer-generated geometries can also be used
to create anatomically realistic in vitro models.
Measurement of velocity at discrete points within
the in vitro models can be obtained by laser Doppler
anemometry or particle image velocimetry, for
optically clear models, or the models can be made
magnetic resonance imaging-compatible and flow
measured by imaging techniques. Pressure transducers
can record pressure at discrete points, though not
throughout the entire flow field. In vitro models have
the advantage that no complex mathematical equations
need to be solved to resolve features of the flow
field, but modeling other system parameters can be
difficult, such as matching the viscous, elastic, and
porous properties of tissue.

The arachnoid trabeculae, nerve roots, and
perivascular vessels of the spinal cord are often
neglected in simulations of cerebrospinal fluid
motion due to the spatial resolution required to
image the fine structures relative to the rest of the
subarachnoid space and the computational complexity
required to incorporate them into model geometries.
Stockman47 conducted cerebrospinal fluid flow
simulations using the lattice Boltzmann method in a
rigid annular model that included fine anatomical
structures of the spinal subarachnoid space such as
trabeculae, nerve bundles, and denticulate ligaments
(Fig. 5). The study found that longitudinally averaged
flow was not significantly affected by the
arachnoid trabeculae when the spacing of trabeculae
was regular. Variation in trabeculae width had little
effect on the velocity profile for a given pressure

distribution, which implied that the pressure environment
in the spinal subarachnoid space was not
greatly affected by fine structures. However, the
arachnoidal trabeculae were found to be important
in the superior cranial subarachnoid space in a
subject-specific simulation by Gupta et al.48 with the
domain modeled as a porous medium. In this
simulation, the trabecular microstructure of the
superior cranial subarachnoid space was found to
offer substantial resistance to cerebrospinal fluid
flow. Doubling of the trabecular density from a
nominal value corresponding to a subarachnoid
space porosity of 0.99 increased the pressure drop
across the cranial subarachnoid space by a factor of
2.5. Large spatial variations were found in the
velocity distribution in the cranial subarachnoid
space, which the authors proposed may influence
transport behavior of metabolites, neuroendocrine,
and other substances in the cranial cerebrospinal
fluid circulation.

Roldan et al.54 simulated
cerebrospinal fluid flow in rigid geometrically
realistic spinal subarachnoid space models based
on magnetic resonance images from the spinal canal
of a patient and a healthy volunteer. The study
employed the boundary element method, which
neglects inertial effects, to solve the Navier–Stokes
equations. Peak systole and peak diastole were
modeled separately as steady flow. The pressure
gradient between the inlet and outlet was found to
be steeper in the patient model than in a healthy
model of the same length. Peak pressures in the
patient model were 1.5 times higher than the healthy
model. Flow fields were heterogeneous with fluid
jets observed anterolaterally to the spinal cord in
both models. Qualitatively these results were consistent
with findings from several phase-contrast
imaging studies.10,25,37 Significant velocity vector
components were observed perpendicular to the
long axis of the spinal canal (Fig. 7). Linge et al.55
produced similar results using a geometrically
idealized model of the posterior cranial fossa and
cervical spine. This study examined the effect of
anatomic variation on cerebrospinal fluid hydrodynamics.
Though the geometry was idealized,
spatial variations in flow patterns resembled those
observed in in vivo phase-contrast imaging studies.
The simulated pressure pulse (0.75 mmHg peak-topeak
amplitude) compared favorably with in vivo
cerebrospinal fluid pulse pressure measurements in
healthy subjects (1–4 mmHg56–58).

At present, the hydrodynamic parameters of interest
are geometry, velocity and volume flow, compliance
and tissue mechanical properties, resistance, and
pressure. The following discussion details each of
these parameters in light of the current findings and
challenges involved in their determination.

Brain settling may cause increases in fluid
velocity and resistance to flow, which in turn creates a
larger pressure gradient that may further alter the
compliance of the craniospinal system,14 and the
cascade may continue.

\subsection{loth 2014}
CFD has been applied in models with varying degrees of ana-
35 tomical complexity to help understand the underlying physiology
36 of CSF flow [17–19]. A review of CSF flow studies using CFD
37 was published by Shaffer et al. [20] and Yiallourou et al. [21].




\section{Dissertation}
Flow models coupled with MRI
flow measurements may become a noninvasive tool to explain the abnormal dynamics of
CSF in related brain disorders as well as to determine concentration and local distribu
tion
of drugs delivered into the CSF space.

Numerical
models can provide estimates of wave speed; however, the difficulty of
determining tissue elasticity properties as well as accurate dimensions
diminishes the reliability of these estimates.

The methodology requires verification
through phantom model studies and optimization of data processing.

serve as a prognostic indicator in conjunction
with geometric magnetic resonance measurements

\subsection{Numerical methods}
The computations were extended until the initial condition effects
were eliminated or periodwise convergence was achieved in
each cycle, and only computations of the last cycle are reported.
One-dimensional grids with 101 grid
nodes and a time step increment of 1 ms were used.
The convergence
criterion was 10-6 percent of the flow rate at each
Newton–Raphson iteration.

\section{theory}

\subsection{wave transmission}

One method to compute the PWV
in the coaxial tube system involves the use of the Moen-Korteweg
equation.



\section{Anatomy}
This fluid cushions the brain from impact on the cranial vault
walls due to sudden motions. CSF also delivers nutrients and protein
to/from brain surface along with the removal of waste products. CSF is produced within the ventricular cavities at an average rate of 0.3 to 0.7 ml/min and is withdrawn mainly into the superior
sagittal sinus (Loth 2001)

CSF volume in the cranio-spinal cavity is estimated
to be 125 ml.(Loth 2001)

Dilute amounts of monoamines and proteins
are present in normal CSF, whereas the pathological cases may
contain high concentrations of red blood cells ~hemorrhage!
and white blood cells ~meningitis!.(Loth 2001)

\subsection{Origin of CSF}
Several mechanisms have been suggested
for the origin of CSF pulsation \cite{bering_circulation_1962} \cite{du_boulay_pulsatile_1966}, the primary one being
changes in intracranial blood volume during the cardiac cycle.
Dunbar et al. \cite{dunbar_study_1966} found the lumbar CSF pressure pulsations to
persist in dogs with the cervical canal blocked, indicating the
presence of a secondary driving mechanism. (Loth 2001)

Recent developments in noninvasive flow measurement techniques,
such as color Doppler ultrasound and phase encoded MRI,
enabled the in vivo measurements of the velocity pulsation. Using
MRI, Feinberg and Mark \cite{feinberg_human_1987} and Enzmann and Pelc \cite{enzmann_cerebrospinal_1993} found
correlation between brain motion and CSF flow waveforms. Winkler
\cite{winkler_cerebrospinal_1994} showed the synchronous nature of the CSF pulsation
with respiration in normal infants, with ultrasound and echography.

CSF velocity waveforms in the ventricular, cisternal, and
cervical subarachnoid spaces were described by Bhadelia et al. \cite{bhadelia_analysis_1995}
and Henry-Feugas et al. \cite{henry-feugeas_temporal_1993}. Alperin et al. \cite{alperin_mri_2005}
used phase contrast MRI of CSF and blood flow to characterize
the CSF dynamics in terms of a hemodynamically independent
CSF transfer function (very interesting). Significant work has been done to measure
CSF and brain tissue motion within the spine and cranium using
MRI.

It has also been observed that the spinal cord moves a small but
detectable amount during CSF pulsation. Motion sensitive MRI
techniques have been used to quantify the motion of the cord as
well. Researchers have found the spinal cord to move within the
spinal canal at a peak velocity of approximately 0.7–1.2 cm/s

\subsection{Data}
CSF is produced within the ventricular cavities at an average rate of 0.3 to 0.7 ml/min (loth 2001)

Researchers have found the spinal cord to move within the
spinal canal at a peak velocity of approximately 0.7–1.2 cm/s (loth 2001)

\section{Introduction}

Several theory of CSF circulation : choroid plexus as a pump, arterial rise in cranial vault, veins compression and this original but somehow theory that when the spinal canal is occluded, CSF remains under pressure => it is like there are several plans and ways for the CSF to move in case of failure of one of them, the huma body is amazing. Talk about the history of the CSF. A l'heure ou l'esprit de transversalité et multidisciplinarité nourrit la recherche scientifique, il me paraît important de souligner les points de vue de disciplines telle que l'ostéopathie sur le LCR.

Cerebrospinal fluid pressure dynamics are difficult
to simulate due to the complexities detailed above.
Even if the pressure boundary conditions for computational
simulations are measured invasively, the
simulated results are suspect due to the necessity to
simplify and decouple different parts of the cerebrospinal
fluid and communicating systems. For
example, decoupling of the spinal and cranial
cerebrospinal fluid systems has been common in the
existing studies and could make anomalies in the
approximated flow field (e.g. seemingly random pressure
or velocity fluctuations) difficult to justify in the
context of only one part of the system.51

\section{Craniospinal disorders}

\cite{oldefield}

For example,
large values of VWS may provide an indication of elevated intracranial
pressure. Chiari I malformation may produce an altered pressure
environment that is detectable by this methodology as well

Type I Chiari malformation has historically been
described as a change in the morphology of the
hindbrain, characterized by herniation of the cerebellar
tonsils past the foramen magnum by 3–5 mm,
as diagnosed by magnetic resonance imaging

The herniation results in reduced crosssectional
area of the subarachnoid space at the
foramen magnum

The present understanding of
the pathophysiological cascade in type I Chiari
malformation follows the following logic:
1. morphological changes to the cerebellum crowd the
subarachnoid space near the foramen magnum;6–9
2. crowding of the local subarachnoid space results in
obstruction of cerebrospinal fluid flow pulsations
at the foramen magnum;
3. the obstruction of flow pulsations results in
abnormal cerebrospinal fluid velocities10,11 and
potentially increased resistance;
4. increased resistance could reduce cerebrospinal
fluid flow between the cranial and spinal subarachnoid
spaces with each cardiac pulsation.
However, the driving pressure (arterial pressure)
is much larger than intracranial pressure and, thus,
can force the same volume of cerebrospinal fluid
out of the cranium even with an obstruction;
5. an increased pressure gradient is required to push
the same volume of cerebrospinal fluid from the
cranial to the spinal subarachnoid space with the
presence of an obstruction;
6. the increased pressure gradient may consequently
displace brain tissue, resulting in further alteration
to the morphology of the cerebellum, and produce abnormal biomechanical forces acting on the
neural tissue and vasculature.

A craniospinal pathology often accompanying type I
Chiari malformation that may also significantly
impact the hydrodynamic environment of the craniospinal
system is syringomyelia. Syringomyelia is
characterized by a fluid filled cyst or cysts that form
in the spinal cord parenchyma that can expand over
time, eventually obstructing cerebrospinal fluid
movement in the spinal subarachnoid space. The
cyst(s), or syrinx(ges), can form caudal to a flow
obstruction, such as the cerebellar tonsil herniation in
type I Chiari malformation, and have direct connection
to the fourth ventricle or be entirely enclosed as
in non-communicating syringomyelia.
From a mechanical perspective, the reason for
syrinx formation and progression is unclear because
(1) the fluid inside a non-communicating syrinx can
be at a considerably higher pressure than the
cerebrospinal fluid in the surrounding subarachnoid
space;59–61 and (2) the cerebrospinal fluid has been
shown to communicate into the syrinx cavity through
the perivascular spaces.62–64 Thus, a major research
question has been how and why fluid moves into and
accumulates in the syrinx cavity. A passive mechanism
for fluid movement into the syrinx alone is
difficult to reconcile mechanically

\section{Geometric}

\subsection{Loth 2001}

The spinal cavity of Visual Man of the Visible Human Project from the
National Library of Medicine  was selected as the source for 
the model geometry. Axial anatomic images (204831216) were
used to identify the spinal cavity. Fifty images ~I.D. 1161–1651!
were selected, which spanned the spinal length at a vertical spacing
of 1 cm. The spinal canal of the Visual Man is a variable cross-sectional annular
space with an overall length of 70 cm and an average outer diameter
of 3 cm, resulting in an aspect ratio ~L:D! of approximately 
24:1. The external boundary is 20 cm longer than the cord, and the
end of the duct is closed. Hence, the lower part of the duct is a
pipelike structure. The cord is connected along the duct to the
external wall via nerve junctions and denticulate ligaments. The
end of the spinal cord is also connected to the duct by a dense
cluster of nerve roots, called the cauda equina \cite{parson}. The end of the canal where the spinal cord divides into a bundle of many smaller nerves was omitted (20 cm).

The outer radius was fixed at 10 mm and the inner radius
changed from 0.5 to 9.0 mm

\begin{figure}[!h]
\centering
\caption{\label{anatomy} Flow waveform obtained through phase contrast MR on a healthy subject}
    \includegraphics[scale=0.3]{loth_2001_hydraulic_diametre}
\end{figure}

\subsection{loth 2011}

Obtaining an accurate representation of the cerebrospinal
fluid system geometry is difficult with the
pre-processing workflow required to perform computational
fluid dynamics simulations. To perform these
simulations, the geometry images need to be segmented
and smoothed to form the numerical geometry
which involves difficult interpretation of the fine and
complex anatomical structures in the cerebrospinal
fluid system contained within the images. At present,
the precision of image-based geometry measurements
are on the order of millimeters with varying levels of
repeatability and accuracy. The dimensions of the
subarachnoid space in a patient with type I Chiari
malformation can be small near the herniation with
complex morphology, which could translate into
significant errors when simulating fluid flow. In
particular, the pressure gradients required to move
cerebrospinal fluid are highly sensitive to dimensions.
For steady flow in straight circular pipe, the pressure
gradient (dP/dz) required to cause flow (Q) is
proportional to the inverse of the diameter (D) of
the pipe to the fourth power along with fluid viscosity
(m) and flow (dP/dz5128mQ/pD4). Gap dimensions
for collagen meniscus implant (CMI) patients can be ;
as small as one millimeter. Errors in these gap
dimensions due to image resolution could easily be
20–50%, which would lead to large errors in pressure
gradient calculations. In addition, hydrodynamic
simulations are typically limited to local regions of
the cranial, cervical, thoracic, or lumbar subarachnoid
space. This is a product of the limitations in
magnet coil size and scanning time.

An added complexity in evaluating craniospinal
geometry is that tissue moves during the cardiac cycle.
At present, it is unclear how influential tissue motion is
on the hydrodynamic environment. This motion has
been reported as small but detectible and may not be
negligible given the importance of gap size. Brain
displacements as measured by phase-contrast imaging
have been described to be 0.1–0.2 mm with velocities
in the range of 1–2 mm/s.13,77,78 In addition, spinal
cord motion has been measured in healthy subjects
and velocity values were even greater (12.4¡2.9
mm/s79 and 7.0¡1.4 mm/s80). Alperin et al.29 reported
maximum spinal cord displacement for control volunteers
and patients with type I Chiari malformation to
be 0.33 and 0.39 mm, respectively.

\section{velocity}

\subsection{loth 2011}

While investigating intracranial compliance,
Alperin et al.29 demonstrated that volume flow analysis
may offer more insight to the altered biomechanical
environment than velocity field analysis. Results
in that study showed that peak volume flow rate
measured at the C2 level was higher in volunteers
(215 ml/min) than in CMI patients (190 ml/min).
However, the net volume of fluid displaced during the
cardiac cycle was similar between the two cases
(0.57 ml healthy versus 0.56 ml patient). This implies
that while increased resistance due to tonsillar
herniation may affect velocity magnitudes throughout
the cardiac cycle, flow rate may not be affected
in the same way. Pressure gradients in the spinal
canal (dP/dz) would then be forced to increase to
maintain volume flow in the presence of increased
resistance. Prolonged pressure elevation may then
affect the elastic properties of the tissue in the
craniospinal system and, thus, change the compliance
of the system.

\section{Flow modelling}

\subsection{loth 2001}


Annular spinal canal, pulsatile flow, spinal cavity (SAS) assumed rigid (the assumption that the flow waveform does not change along the spine ~i.e., rigid wall assumption), zero resistance expansion of the spinal cavity end.
Simulations of the flow field within the spinal cavity were conducted
using a one-dimensional model for a circular annulus to
understand the basic gross features. A two-dimensional formulation
was used to examine effects of eccentricity and noncircular
cross sections.
One-dimensional model : annular spinal canal with two cases, one moving and stationary cord.
Entry data : CSF flow waveform obtained in vivo using dynamic phase contrast MRI.

Two-Dimensional Model (Elliptic Spinal Cavity and Circular
Spinal Cord). as eccentricity and noncircular boundaries

In these idealized models, effects of nerve roots and wall compliance
were not taken into account.

\section{MRI}
\subsection{loth 2001}
Dynamic phase contrast magnetic resonance ~MR! was used to
~
measure the CSF pulsation in a healthy adult volunteer at the level
of the second cervical vertebra. The MRI technique used to ac
quire
these images was previously reported by Alperin et al.
The CSF flow waveform shown in Fig. 4 was computed by inte
gration
of the velocity values within the CSF flow area ~lumen!.

The mean component of the flow waveform is nearly zero, depict
ing
an oscillatory motion inside the spinal cavity with a peak flo
rate of 360 ml/min. The axial variation in velocity is expected to
be significant along the spinal canal due to area changes. Based on
the peak flow rate and the assumption that the flow waveform
remains the same along the spine, the average velocity ~spatially
~
averaged across the cross section! would range from 1–8 cm/s
with the maximum value located approximately 20 cm below the
skull.

\subsection{loth 2009}

It is important to select the
pixels near the center of the gap since a phase shift can occur for velocity
traces at different distances from the wall. This is a well-documented
phenomenon for pulsatile (Womersley) flows and would add noise to
the calculation of VWS.

\subsection{loth 2014}
MRI data acquisition was performed as in article "The Impact of Spinal Cord
 Nerve Roots and Denticulate Ligaments on Cerebrospinal Fluid Dynamics in
 the Cervical Spine."


\begin{figure}[!h]
\centering
\caption{\label{anatomy} Flow waveform obtained through phase contrast MR on a healthy subject}
    \includegraphics[scale=0.3]{loth_2001_flow_waveform}
\end{figure}

\subsection{loth 2011}
These
tools include one-dimensional through-plane phase
contrast magnetic resonance imaging for measurement
of cerebrospinal fluid flow,10,24,25 high temporal
resolution in-plane cerebrospinal fluid pulse wave
velocity measurement,26 measurement of cerebellar
tonsil movement,13,27 and quantification of an index
of craniospinal compliance.28,2
These modalities include techniques
to measure cerebrospinal fluid velocity in 7D,30
magnetic resonance diffusion tensor imaging to
measure neural fiber tract direction and alignment31
and brain cerebrospinal fluid content and diffusion
properties32, magnetic resonance elastography to
measure brain elasticity,33,34 and magnetic resonance
spectroscopy to measure levels of different metabolites
in the brain

Of the dynamic magnetic resonance imaging techniques
mentioned above, phase-contrast imaging has
been the most widely explored, as it provides in vivo
measurement of velocity that can be valuable to
understanding the flow environment

Cerebrospinal fluid velocity wave speed in the
spinal subarachnoid space of three healthy subjects
was calculated to be 4.6 m/s by Kalata et al.26 The
study used a novel high-speed sagittal in-plane
pcMRI measurement technique. As velocity wave
speed is known to relate to the material properties of
a flow conduit, it may provide an estimate of tissue
stiffness in the craniospinal system when measured in
the spinal subarachnoid space.

Phase-contrast imaging-based velocity measurements
may also have significant error and could be
improved in many ways. Phase-contrast imagebased
measurements are limited to velocity in a
single direction (i.e. through-plane or in-plane
velocity in a single direction) at approximately 30
time points during the cardiac cycle. The main
sources of error are from noise, breathing artifacts,
and difficulty in selection of velocity encoding value
since cerebrospinal fluid velocities may vary widely.
Difficult regions for fluid velocity measurement are
the spinal subarachnoid space and lateral ventricles
where cerebrospinal fluid velocities are particularly
low and the influence on fluid motion from breathing
is maximal. Cerebrospinal fluid velocities are
also difficult to measure in regions with complex
flow patterns when significant portions of the
velocity are not in the direction of velocity
encoding, such as at the foramen magnum in type
I Chiari malformation. In these measurements,
integration of velocity to determine hydrodynamic
parameters such as flow volume can also introduce
error since the region of interest in the subarachnoid
space cross-section needs to be interpreted. In
the context of type I Chiari malformation, velocity
can be the greatest in the narrow regions and thus
the region of interest selection can have a critical
impact. Techniques that may help improve fluid
velocity measurement, and thereby calculation of
hydrodynamic parameters from phase-contrast
images, include reduction of signal noise from
breathing, automatic optimization of velocity encoding
values, better selection and optimization of
the region of interest for flow measurement, greater
temporal resolution, and velocity measurement in
multiple directions within an entire volume of
cerebrospinal fluid.

\section{compliance}

\subsection{loth 2011}

It has been hypothesized that, under normal
conditions, the healthy spinal subarachnoid space
could act as a sort of notch filter (coupe bande) to dampen
incoming cerebral blood flow pulsations to supply
smooth blood flow to the neural tissue by Madsen
et al.,81 Luciano and Dombrowski,82 and others

Thus, any disruption to the system that alters
compliance, such as an obstruction to cerebrospinal
fluid flow, could reduce the damping effect on
cerebral blood flow pulsations. A reduction in
damping of the cerebral blood flow pulsations
would result in abnormal biomechanical forces
acting within the craniospinal, arterial, or venous
system.

Non-invasive
methods of measuring compliance and tissue
mechanical properties in the spinal subarachnoid
space are being developed. These include magnetic
resonance imaging techniques that allow calculation
of overall compliance in the spinal subarachnoid
space from cerebrospinal fluid velocity wave speed
in the spinal subarachnoid space26 and magnetic
resonance elastography to measure brain elasticity
33,34 and local material properties.

A major reason for the focus on non-invasive
compliance measurement methods is that there are
many complexities to physically obtaining and
measuring material properties of tissues ex vivo that
may affect compliance assessment of the craniospinal
system. Some of these complexities include (1)
differences in material testing techniques can produce
varying results; (2) testing direction and orientation
can have a large impact on measurements of
anisotropic tissues

and (3) removal and separation
of each tissue component is not straight forward,
easily repeatable, or always complete

In addition,
the time after harvesting, subject age, and preservation
methods may influence tissue properties.

While resistance can be increased by changes in
geometry such as cerebellar tonsil herniation, the
impact on cerebrospinal fluid hydrodynamics can
follow two different scenarios. First, if the pressure
gradient in the subarachnoid space increases greatly,
the cerebrospinal fluid flow rate (i.e. the volume of
fluid leaving the cranium) can be maintained.
However, if the pressure gradient in the subarachnoid
space remains unchanged, the flow rate would
decrease correspondingly. Phase-contrast imaging
measurements of velocity could be greater or smaller
for an obstructed versus an unobstructed subarachnoid
space due to the two possible scenarios as well
as velocity jetting in the obstructed subarachnoid
space.

\section{pressure}

\subsection{loth 2011}

In vivo pressure measurements indicate that pressure
magnitude and gradients have an impact in type I
Chiari malformation in terms of symptoms and
severity. While magnetic resonance imaging methods
have provided information about velocity and
geometry of the cerebrospinal fluid system, they are
unable to measure pressure. Invasive measurements
of pressure are possible but require creation of an
access point to the subarachnoid space which alters
the system and may not permit accurate measurements.
Nevertheless, cerebrospinal fluid pressure has
been quantified in a limited number of invasive
studies to be 7–15 mmHg in the supine position and
0–10 mmHg in the vertical position in healthy
subjects.90,91

Pressure in healthy subjects and patients
with type I Chiari malformation has been measured
in a number of ways including craniospinal pressure
dissociation, which is obtained by measuring instantaneous
pressure differences between ventricular and
lumbar cerebrospinal fluid pressure, a technique
introduced by Williams.56,92 Williams’ measurements
indicated that pressure differences between the
ventricles and spinal subarachnoid space are greater
in patients with type I Chiari malformation than in
healthy subjects. In another study by Sansur et al.,93
it was found that cerebrospinal fluid pressure
measured during coughing was elevated in patients
with headache in comparison to patients without
headache and healthy volunteers.

Pressure gradients in the cerebrospinal fluid system
are the driving forces that cause tissue and cerebrospinal
fluid motion and may be the cause for
nerve damage in type I Chiari malformation.14 Local
cerebrospinal fluid pressure magnitude could also
cause damage to the neural tissue by disrupting the
normal flow of blood, interstitial, and/or lymphatic
fluid within the tissues. Thus, a detailed understanding
of the pressure within the cerebrospinal
fluid, blood, interstitial, and lymphatic fluid would be
helpful toward understanding the pathophysiology of
type I Chiari malformation and related craniospinal
disorders such as syringomyelia.

Many structural and communicating factors influence
cerebrospinal fluid system pressure dynamics.
Structural factors include the structural layers and
neural tissues, such as the vertebrae, skull, brain,
spinal cord, dura, pia, and arachnoid membrane,
that each have complicated anisotropic, nonlinear,
and poroviscoelastic properties. The cerebrospinal
fluid system communicates with the cardiovascular
system through the veins and arteries supplying
blood to the neural tissue.84,94 In particular, pressure
in the venous system likely has a great impact on
cerebrospinal fluid pressure since pressure in the
venous vascular bed is normally only slightly lower
(1–3 mmHg) than in the cerebrospinal fluid, with the
veins only held from collapsing by their structural
rigidity.95 Communication between the cerebrospinal
fluid and intrathoracic pressure due to postural
changes,96,97 coughing,93 valsalva and Queckenstedt’s
test, and abdominal pressure98 has been well
documented.56,61,92,93

Cerebrospinal fluid pressure dynamics are difficult
to simulate due to the complexities detailed above.
Even if the pressure boundary conditions for computational
simulations are measured invasively, the
simulated results are suspect due to the necessity to
simplify and decouple different parts of the cerebrospinal
fluid and communicating systems. For
example, decoupling of the spinal and cranial
cerebrospinal fluid systems has been common in the
existing studies and could make anomalies in the
approximated flow field (e.g. seemingly random pressure
or velocity fluctuations) difficult to justify in the
context of only one part of the system.


\section{Flow characteristics}
\subsection{loth 2001}
Based on these velocities and the hydraulic diameter, the
instantaneous Reynolds number ~Re! at peak flow was found to
vary between 150 and 450 with the maximum located at the base
of the spine. Along the spine, the Womersley number
was shown to vary between 5 and 18 with the
largest values near the skull and base of the spine. The Womersley
number was based on hydraulic radius, viscosity, and the cardiac
cycle frequency ~50 beats per minute!.

\section{Spinal cord}

\subsection{loth 2001}
The importance of cord movement was assessed by computing
the CSF pulsation for a moving cord ~radius 6.5 mm! based on the

MR measurement of cord motion. Spinal cord motion produced
significant changes in the hydrodynamics for inner radii greater
than 8.0 mm. For inner radii less than 8.0 mm, the pressure gra
dient
and shear stress waveforms remained unaltered, and varia
tions
in velocity profile were constrained to the region near the
cord. The pressure gradient waveform also remained unaltered
due to the inertial character of the flow

\section{Results}

\subsection{loth 2001}

Peak velocity was largest near the middle of the spinal cord due to area changes
with a maximum average value of 8 cm/s
Based on these parameters
inertial effects are expected to dominate the flow field for normal
physiological flow rates and CSF fluid properties; especially in the
cervical and lower lumbar regions where Womersley numbers
were largest.
Flow is expected to remain laminar throughout the
cycle as the Reynolds number remains sub-critical
Recritical
,2100) throughout the cycle.
Womersley numbers presented in this study may be slightly lower than average since the values were based on a subject’s
CSF flow waveform with a heart rate of 50 beats per
minute, which is lower than average.

For very small annular gaps ~or large values of inner radius, the
viscous effects dominate the flow and the pressure gradient wave
forms begin to resemble the CSF input flow waveform shown in
Fig. 4. For large annular gaps, i.e., high Womersley numbers,
inertial effects dominate.

Observed cross-sectional areas were largest in the cervical
and lower lumbar regions of the spine, again indicating inertia
dominated character of CSF flow in these regions.

The results resemble classical Womersley
flow profiles in circular pipes. For moderate to high Womer
sleynumbers Figs. 8, 11, and 12!, the velocity profile is blunt
compared to a parabolic shaped profile at low Womersley num
bers ~Fig. 9!. The inertia dominated cases also generated hornlike
structures near the wall boundaries during systole and diastole.

The MR flow measurements taken at the base of the neck were
assumed to be the same for the entire length of the spine. This
assumption implies no compliance of spinal cavity boundaries.
We expect compliance effects to be small in the neck and thoracic
region and significant at lumbar region. Thus, our calculations of
nondimensional parameters in the neck and thoracic regions
should provide good estimates of the in vivo values but may overestimate
those in the lumbar region. Further studies of the CSF
flow waveform distribution along the spinal canal must be conducted
to confirm this.


\subsection{loth 2009}

A number of studies have reported a single value of VWS during
the CSF flow cycle. VWS in the CSF was found to be 13.5 m/s
by Williams [2], 2.2–4 m/s by Carpenter et al. [3], 4 m/s by Greitz
et al. [4], and 12.4 m/s by Bertram et al. [12]. In an in vitro study of
syringomyelia by Martin et al., the VWS was found to vary during the
CSF flow cycle from 2 to 26 m/s

During diastole, the flow is reversed and the pressure in the CSF system
of the spine is lower. Due to the nonlinear stiffening properties of the
spinal tissues, the transient state of lower pressure results in a higher
system compliance during diastole, which manifests itself in a lower
wave speed through the system. However, the results during diastole in
this study were not statistically significant, and thus, we cannot confir
VWS variation during cardiac cycle. Statistically significant data were
obtained only for the wave traveling caudally using the maximum
velocity gradient (during systole) as a time point marker.

The first measurement of wave
speed was byWilliams in 1976 via direct puncture into the lumbar and
cervical spinal canals.

the lumbar and cisternal regions, which means that the assumption
that the wave traveled from the lumbar to cervical region may not
be correct [3]. Thus, the distance traveled was potentially shorter than
assumed, whichwould mean that thewave speedwas likely smaller than
reported, and in Carpenter et al.’s assessment, “.
the true value lay in
. .
the range of 4 to 5 m/s” [3]. This value is in agreement with Carpenter
et al.’s computed value from numerical simulations. In 2005, Bertram
et al. also developed a numerical model of the spinal fluid system
and computed wave speed to be approximately 12 m/s [12]. However,
Carpenter et al. and Bertram et al. assumed elastic modulus values and
dimensions for the various tissues, which will likely have a significan
impact on the computed CSF VWS magnitude.
Using an in vitro model representative of syringomyelia, Martin
et al. demonstrated a variation in wave speed through the cardiac cycle
(2–26 m/s) [5]. Because of the linear stiffening of the physical model’s
spinal cord, this VWS variation during the CSF flow cycle could be
different in vivo. This study computed the wave speed within a model
of the spinal SAS in a two-part experiment. First, MR was used to
measure velocity and compute the flow waveform in the model. In a
separate experiment, unsteady pressure measurements were obtained
in the same model in the laboratory. These values were then used to
compute the wave speed employing a modified form of the Moens–
Korteweg equation. In 1999, Greitz et al. reported the CSF VWS of
approximately 4 m/s [4]. This value matches the values from the nu
merical
studies [3], [12]; however, details of how Greitz et al. obtained
this value are not fully explained.
Phase lag was recorded after the subject was
asked to cough, thereby obtaining an approximatewave speed of 13 m/s.
Carpenter et al. noted that the pressure pulse likely originated between

the lumbar and cisternal regions, which means that the assumption
that the wave traveled from the lumbar to cervical region may not
be correct [3]. Thus, the distance traveled was potentially shorter than
assumed, whichwould mean that thewave speedwas likely smaller than
reported, and in Carpenter et al.’s assessment, “.
the true value lay in
. .
the range of 4 to 5 m/s” [3]. This value is in agreement with Carpenter
et al.’s computed value from numerical simulations. In 2005, Bertram
et al. also developed a numerical model of the spinal fluid system
and computed wave speed to be approximately 12 m/s [12]. However,
Carpenter et al. and Bertram et al. assumed elastic modulus values and
dimensions for the various tissues, which will likely have a significan
impact on the computed CSF VWS magnitude.
Using an in vitro model representative of syringomyelia, Martin
et al. demonstrated a variation in wave speed through the cardiac cycle
(2–26 m/s) [5]. Because of the linear stiffening of the physical model’s
spinal cord, this VWS variation during the CSF flow cycle could be
different in vivo. This study computed the wave speed within a model
of the spinal SAS in a two-part experiment. First, MR was used to
measure velocity and compute the flow waveform in the model. In a
separate experiment, unsteady pressure measurements were obtained
in the same model in the laboratory. These values were then used to
compute the wave speed employing a modified form of the Moens–
Korteweg equation. In 1999, Greitz et al. reported the CSF VWS of
approximately 4 m/s [4]. This value matches the values from the nu
merical
studies [3], [12]; however, details of how Greitz et al. obtained
this value are not fully explained.
Phase lag was recorded after the subject was
asked to cough, thereby obtaining an approximatewave speed of 13 m/s.
Carpenter et al. noted that the pressure pulse likely originated between

\begin{figure}[!h]
\centering
\caption{\label{anatomy} Velocity Wave Speed (VWS)}
    \includegraphics[scale=0.1]{loth_2009_vws}
\end{figure}

\section{Discussion}

\subsection{loth 2011}

Currently, the state of type I Chiari malformation
research is developing magnetic resonance imaging
protocols for direct measurement of hydrodynamic
parameters by imaging methods or indirect hydrodynamic
parameter calculation through computational
models. The goal of examining these
parameters is to provide clinically useful information
to improve care and treatment for patients with type I
Chiari malformation. Some challenges to clinical
translation of direct or indirect hydrodynamic parameters
include:
hydrodynamic parameters are not well-established.
As with all biological data, each hydrodynamic
parameter could vary significantly with age, sex,
weight, and other factors. Thus, establishing indices
for normal versus pathological hydrodynamic
parameters is a difficult task. One possible workaround
could be to develop parameters that are
based on subject specific diagnostic tests rather
than direct comparison to healthy subjects. These
tests could be performed on a case by case basis to
examine how the cerebrospinal fluid system
responds to a particular stimulus. This type of test
would be analogous to diagnostic techniques for
assessment of stroke, coronary artery disease, and
hypertension by vasodilation;75

computational models require many assumptions.
The assumptions will introduce error to the
parameters calculated by the models. Some
assumptions are close to reality, such as assuming
cerebrospinal fluid behaves like water.76 However,
other assumptions such as rigid, impermeable
conduit boundaries and homogenous tissues could
be an oversimplification. It is difficult to conclude
when and which assumptions are valid, as these
measurements are difficult to make in vivo;

magnetic resonance imaging measurements have
resolution limits. It is possible that the current
imaging limitations are one of the confounding
factors behind seemingly contradictory cerebrospinal
fluid hydrodynamic studies. In addition, the
accuracy of the simulations can only be as good as
the boundary conditions used. Thus, it is important
to perform boundary condition sensitivity analysis
on the results of computational fluid dynamics
simulations before making conclusions on the
hydrodynamics;
4. parameter interpretation is complex. While magnetic
resonance imaging and simulations have
provided many hydrodynamic parameters for
assessment of type I Chiari malformation, better
fundamental understanding of the cerebrospinal
fluid system is needed to correctly interpret what
influence these parameters have on the global
dynamics. Additional complexities are also
involved in correlation of parameters with clinical
results such as symptom improvement, which is
highly subjective, but also most necessary.

Careful examination of hydrodynamics in type I
Chiari malformation offers potential for better
understanding of pathophysiology and clinical utility.
Key parameters are geometry, velocity, compliance,
resistance, and pressure. It is unclear which parameter
is most important and it is likely that a
combination of parameters is necessary to assess a
pathological state. Studies of hydrodynamics in type I
Chiari malformation are sparse as yet, with the
exception of clinical phase-contrast imaging studies
of cerebrospinal fluid velocity. Engineering-based
models may help identify parameters that could be
evaluated to assess clinical significance. This may
assist current research efforts that are focused on
developing magnetic resonance imaging protocols
with an eye toward clinical applications.

\section{dental and roots ligaments}
The Impact of Spinal Cord
 Nerve Roots and Denticulate Ligaments on Cerebrospinal Fluid Dynamics in
 the Cervical Spine. This study showed that NRDL has an important impact on
41 CSF dynamics in terms of velocity field and flow patterns.
average thickness of nerve rootlets and den- 100
ticulate ligaments were 1.4 and 0.1 mm, respectively

its anatomy in litterature
Alleyne, C. H., Jr., Cawley, C. M., Barrow, D. L., and Bonner, G. D., 1998,
564 “Microsurgical Anatomy of the Dorsal Cervical Nerve Roots and the Cervical Dor-
565 sal Root Ganglion/Ventral Root Complexes,
Lang, J., 1993, Clinical Anatomy of the Cervical Spine, G. Thieme Verlag, Uni-
566 versity of Wurzburg, Germany
Tubbs, R. S., Salter, G., Grabb, P. A., and Oakes, W. J., 2001, “The Denticulate
567 Ligament: Anatomy and Functional Significance,” J. Neurosurg., 94(2 Suppl),
568 pp. 271–275.

\section{blood spinal}
\subsection{loth 2014}


\section{notes}
nous venonsde voir que la frequence de battement cardiaques a un effet sur la pression et le débit du LCS. Cet paramètre sera à prendre en compte lorsqui'il s'agira de comparer les résultats du modèle à ceux de la littérature.
\bibliographystyle{unsrt}
\bibliography{mybibfile}

\end{document}